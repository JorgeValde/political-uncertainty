\documentclass[12pt]{article}

\addtolength{\textwidth}{1.3in}
\addtolength{\oddsidemargin}{-.65in} %left margin
\addtolength{\evensidemargin}{-.65in}
\setlength{\textheight}{9in}
\setlength{\topmargin}{-.5in}
\setlength{\headheight}{0.0in}
\setlength{\footskip}{.375in}
\renewcommand{\baselinestretch}{1.0}
\setlength{\parindent}{0pt}
\linespread{1.1}

\usepackage[pdftex,
bookmarks=true,
bookmarksnumbered=false,
pdfview=fitH,
bookmarksopen=true]{hyperref}

\usepackage[usenames,dvipsnames]{color}
\usepackage{cite}
\usepackage{times, verbatim,bm,pifont}


\usepackage{amsbsy,amssymb, amsmath, amsthm, MnSymbol,bbding}

\setcounter{secnumdepth}{-1} 

\newtheorem{definition}{Definition}
\newtheorem{theorem}{Theorem}
\newtheorem{lemma}{Lemma}
\newtheorem{corollary}{Corollary}
\newtheorem{assumption}{Assumption}
\newtheorem{fact}{Fact}
\newtheorem{result}{Result}

\newcommand{\ve}{\varepsilon}
\newcommand{\ov}{\overline}
\newcommand{\un}{\underline}
\newcommand{\ta}{\theta}
\newcommand{\expect}{\mathbb{E}}
\newcommand{\ga}{\gamma}

\begin{document}

\title{\vskip-0.6in \Large Political Uncertainty}
\author{Kristy Buzard}
\date{\vskip-.1in \today}
\maketitle

On Feb. 12, Sebastian and I agreed to focus efforts on finding a base model to facilitate empirical identification. I am pursuing Groseclose $\&$ Snyder (1996), ``Buying Supermajorities,'' APSR
\begin{itemize}
	\item For each legislator $i$, $v(i) = u_i(x) - u_i(s)$, measured in money; call this the reservation price of $i$
		\begin{itemize}
			\item WLOG, label legislators so that $v(i)$ is a non-increasing function 
			\item Note legislators only have preferences over how they vote, not over which alternative wins
		\end{itemize}
	\item There are two vote buyers; each prefers to minimize total bribes paid while passing his preferred policy, but each would prefer to concede the issue rather than pay more than his WTP
		\begin{itemize}
			\item $A$ prefers $x$; $W_A$ is $A$'s willingness to pay (WTP) for $x$ measured in money
			\item $B$ prefers $s$; $W_B$ is $B$'s WTP for $s$
		\end{itemize}
	\item Bribe offer functions: $a(i)$ and $b(i)$ are $A$ and $B$'s offers to $i$. Legislators take these bribe offers as given and then vote for the alternative that maximizes their payoff
	\item A moves first; $a(i)$ perfectly observable to $B$ when he moves.
	\item Goal: characterize SPNE in pure strategies
		\begin{itemize}
			\item Assume unbribed legislators who are indifferent vote for $s$; all bribed legislators who are indifferent vote for whoever bribed them last
		\end{itemize}
	\item Assume continuum of legislators on $\left[-\frac{1}{2},\frac{1}{2}\right]$
	\item Assume $W_A$ large enough that $x$ wins in equilibrium (no uncertainty case)
	\item $m + \frac{1}{2}$ is the fraction of legislators who vote for $x$, the new policy (as opposed to the status quo, $s$)
	\item Prop 1: three types of equilibria in which $x$ wins; depend on size of $W_B$
	\item Prop 2: $m^*$ (the optimal coalition size) is unique, and provides three cases parameterizing its size in terms of $W_B$, $v(-\frac{1}{2})$ and $v(m^*)$
	\item Prop 3/4: special case where $v(z) = \alpha - \beta v$
\end{itemize}

\vskip.2in
\begin{itemize}
	\item I think, without uncertainty, you would estimate $m^*$ as a function of the parameters of $v$ and WTP
		\begin{itemize}
			\item It's useful that $m^*$ is unique. Not clear it would extend to case of uncertainty
		\end{itemize}
	\item I'm pretty sure all this predicts that $B$ should never pay anything.
		\begin{itemize}
			\item Uncertainty should reverse this, right?
			\item What is uncertainty? I think just make $v(z)$ stochastic
		\end{itemize}
\end{itemize}

\end{document}