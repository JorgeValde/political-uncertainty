\documentclass[12pt]{article}

\addtolength{\textwidth}{1.3in}
\addtolength{\oddsidemargin}{-.65in} %left margin
\addtolength{\evensidemargin}{-.65in}
\setlength{\textheight}{9in}
\setlength{\topmargin}{-.5in}
\setlength{\headheight}{0.0in}
\setlength{\footskip}{.375in}
\renewcommand{\baselinestretch}{1.0}
\setlength{\parindent}{0pt}
\linespread{1.1}

\usepackage[pdftex,
bookmarks=true,
bookmarksnumbered=false,
pdfview=fitH,
bookmarksopen=true]{hyperref}

\usepackage[usenames,dvipsnames]{color}
\usepackage{cite}
\usepackage{times, verbatim,bm,pifont,bbm}


\usepackage{amsbsy,amssymb, amsmath, amsthm, MnSymbol,bbding}

\setcounter{secnumdepth}{-1} 

\newtheorem{definition}{Definition}
\newtheorem{theorem}{Theorem}
\newtheorem{lemma}{Lemma}
\newtheorem{corollary}{Corollary}
\newtheorem{assumption}{Assumption}
\newtheorem{fact}{Fact}
\newtheorem{result}{Result}

\newcommand{\ve}{\varepsilon}
\newcommand{\ov}{\overline}
\newcommand{\un}{\underline}
\newcommand{\ta}{\theta}
\newcommand{\expect}{\mathbb{E}}
\newcommand{\ga}{\gamma}

\begin{document}

\title{\vskip-0.6in \Large Political Uncertainty}
\author{Kristy Buzard}
\date{\vskip-.1in \today}
\maketitle

On Feb. 12, Sebastian and I agreed to focus efforts on finding a base model to facilitate empirical identification. I am pursuing Groseclose $\&$ Snyder (1996), ``Buying Supermajorities,'' APSR
\begin{itemize}
	\item For each legislator $i$, $v(i) = u_i(x) - u_i(s)$, measured in money; this is the reservation price of $i$
		\begin{itemize}
			\item $x$ is an alternative policy proposal; $s$ is the status quo
			\item WLOG, label legislators so that $v(i)$ is a non-increasing function 
			\item Note legislators only have preferences over how they vote, not over which alternative wins
		\end{itemize}
	\item There are two vote buyers; each prefers to minimize total bribes paid while passing his preferred policy, but each would prefer to concede the issue rather than pay more than his WTP
		\begin{itemize}
			\item $A$ prefers $x$; $W_A$ is $A$'s willingness to pay (WTP) for $x$ measured in money
			\item $B$ prefers $s$; $W_B$ is $B$'s WTP for $s$
		\end{itemize}
	\item Bribe offer functions: $a(i)$ and $b(i)$ are $A$ and $B$'s offers to $i$. Legislators take these bribe offers as given and then vote for the alternative that maximizes their payoff
	\item A moves first; $a(i)$ is perfectly observable to $B$ when he moves
	\item Goal: characterize SPNE in pure strategies
		\begin{itemize}
			\item Assume unbribed legislators who are indifferent vote for $s$; all bribed legislators who are indifferent vote for whoever bribed them last
		\end{itemize}
	\item Assume continuum of legislators on $\left[-\frac{1}{2},\frac{1}{2}\right]$
	\item Assume $W_A$ large enough that $x$ wins in equilibrium (no uncertainty case)
	\item $m + \frac{1}{2}$ is fraction of legislators who vote for $x$ as opposed to the status quo, $s$
	\item Results
	\begin{itemize}
		\item Prop 1: three types of equilibria in which $x$ wins; depend on size of $W_B$
		\item Prop 2: $m^*$ (the optimal coalition size) is unique, and provides three cases parameterizing its size in terms of $W_B$, $v(-\frac{1}{2})$ and $v(m^*)$
		\item Prop 3/4: special case where $v(z) = \alpha - \beta z$
	\end{itemize}
\end{itemize}

\vskip.3in
General thoughts on extension to uncertainty
\begin{itemize}
	\item I think, without uncertainty, you would estimate $m^*$ as a function of the parameters of $v$ and WTP
		\begin{itemize}
			\item It's useful that $m^*$ is unique. Not clear it would extend to case of uncertainty, but I think it's likely so I'm going to assume it for now
		\end{itemize}
	\item I'm pretty sure  this predicts that $B$ should never pay anything when there is no uncertainty, but I don't see where they say it explicitly (I should read more carefully to verify)
		\begin{itemize}
			\item Uncertainty should reverse this, right?
			\item What is uncertainty? Make $v(z)$ stochastic is most natural
				\begin{itemize}
					\item I'm going to start with linear parameterization of $v(z)$ and add uncertainty:
						\[
						  v(z) = \alpha -\beta \cdot z + \ve_z
						\]
						we can decide later on the distributional assumptions for $\ve_z$, and whether / when we want to make the $\ve$ vary by legislator ($z$)
				\end{itemize}
		\end{itemize}
\end{itemize}

\vskip.3in
Backward induction (legislature moves last; B makes last bribe; A makes first bribe)
\begin{enumerate}
	\item Legislature
		\begin{itemize}
			\item Each legislator $z$ will decide whether to vote for $x$ or $s$ given $\ve_z$, $a(z)$ and $b(z)$. Votes for $x$ if
			  \[
				  v(z) = \alpha -\beta \cdot z + \ve_z + a(z) + b(z) > 0
				\]
				(whether the inequality is weak or strict depends on tie-breaking rules set out in the paper)
			\item Payoff for vote buyer A if $x$ wins is $U_A(x) - \int_{-\frac{1}{2}}^\frac{1}{2} a(j) dj$
			\item Payoff for vote buyer A if $s$ wins is $U_A(s) - \int_{-\frac{1}{2}}^\frac{1}{2} a(j) dj$
			\item I think the cleanest way to write the condition for whether $x$ wins is
				\[
				  \int_{-\frac{1}{2}}^\frac{1}{2} \mathbbm{1}\left[v(j) \geq 0 \right] dj \geq \frac{1}{2}
				\]
		\end{itemize}
	\item Vote buyer B
		\begin{itemize}
			\item GS assumption on vote buyers' objective is ``each prefers to minimize total bribes paid while passing his preferred policy, but each would prefer to concede the issue rather than pay more than his WTP''
			\item This has to be adapted to passing his preferred policy in expectation? Do we do something like they each have $\frac{1}{2}$ probability of winning? (as long as WTP constraint is satisfied)
			\item Groseclose and Snyder formulation would suggest something like
				\[
				  \min_{b(z)} \int_{-\frac{1}{2}}^\frac{1}{2} b(j) dj \hskip.2in \text{subject to} \hskip.2in \int_{-\frac{1}{2}}^\frac{1}{2} b(j) dj \leq U_B(s) - U_B(x)
				\]
				\[
				\text{and however we write the ``while passing his preferred policy'' constraint}
				\]
		
			\begin{itemize}
					\item With asymmetric $v(z)$ or WTP parameters, it would be easy to get equilibria where only $A$ or $B$ buys votes. But we also have lots of outcomes where both buy votes.
					\item They can easily both have positive probability of winning. But what do we need for this to be an equilibrium in this three stage game?
					\item Uncertainty buys us a lot: no longer this knife edge condition of $A$ pushing to the point that $B$ buys no votes
					\item Maybe it's a much simpler maximize probability of winning net of bribes?
						\[
							\max_{b(j)} \ \left\{ \left[ \int_{-\frac{1}{2}}^\frac{1}{2} \mathbbm{1}\left[\alpha -\beta \cdot j + \ve_j + a(j) + b(j) \geq 0 \right] dj \right]\geq \frac{1}{2} \right\} - \int_{-\frac{1}{2}}^\frac{1}{2} b(j) dj
						\]

				\end{itemize}
		\end{itemize}	
	\item Vote buyer A
		\begin{itemize}
			\item Whatever we decide for vote buyer B will be the same for vote buyer A
		\end{itemize}
\end{enumerate}

\end{document}