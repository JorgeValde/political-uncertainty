\documentclass[12pt]{article}

\addtolength{\textwidth}{1.3in}
\addtolength{\oddsidemargin}{-.65in} %left margin
\addtolength{\evensidemargin}{-.65in}
\setlength{\textheight}{9in}
\setlength{\topmargin}{-.5in}
\setlength{\headheight}{0.0in}
\setlength{\footskip}{.375in}
\renewcommand{\baselinestretch}{1.0}
\renewcommand{\thesection}{\Roman{section}} 
\renewcommand{\thesubsection}{\thesection.\Roman{subsection}}
\linespread{1.5}

\usepackage[pdftex,
bookmarks=true,
bookmarksnumbered=false,
pdfview=fitH,
bookmarksopen=true]{hyperref}
\usepackage[pdftex]{graphicx}
%\usepackage{pdfpages}

%\usepackage[usenames,dvipsnames]{color}
\usepackage{cite}
\usepackage{times, verbatim,bm,pifont,pdfsync}
%\usepackage[hang,flushmargin]{footmisc}%unindents footnotes

% disables chapter, section and subsection numbering
%\setcounter{secnumdepth}{-1} 

\usepackage{amsbsy,amssymb, amsmath, amsthm, MnSymbol,bbding}
\usepackage[hang,flushmargin]{footmisc} 

\newtheorem{definition}{Definition}
\newtheorem{theorem}{Theorem}
\newtheorem{lemma}{Lemma}
\newtheorem{corollary}{Corollary}
\newtheorem{assumption}{Assumption}
\newtheorem{fact}{Fact}
\newtheorem{result}{Result}

\newcommand{\ve}{\theta}
\newcommand{\ta}{\theta}
\newcommand{\ov}{\overline}
\newcommand{\un}{\underline}
\newcommand{\al}{\alpha}
\newcommand{\Ta}{\Theta}
\newcommand{\expect}{\mathbb{E}}
\newcommand{\Bt}{B(\bm{\tau^a})}
\newcommand{\bta}{\bm{\tau^a}}
\newcommand{\bte}{\bm{\tau^E}}
\newcommand{\btn}{\bm{\tau^n}}
\newcommand{\ga}{\gamma}


\begin{document}
\title{\vskip-0.6in LOBBYING AND LEGISLATIVE UNCERTAINTY}
\author{Kristy Buzard\thanks{Syracuse University, Economics Department, 110 Eggers Hall, Syracuse, NY 13244. Ph: 315-443-4079. Fax: 315-443-3717. Email: kbuzard@syr.edu. http://faculty.maxwell.syr.edu/kbuzard.} \and Sebastian Saiegh\thanks{University of California, San Diego, Department of Political Science, Social Sciences Building 365, 9500 Gilman Drive, La Jolla, CA 92093. Ph: 858-534-7237. Fax: 858-534-7130. Email: ssaiegh@ucsd.edu. http://pages.ucsd.edu/~ssaiegh/.}} 
\date{\vskip-.1in \today}
\maketitle

%\begin{center} {\bf Abstract} \end{center}

%\begin{quote}
%{\small This paper 

%\textit{JEL classification:} D72, D80 \\
%\textit{Keywords:} lobbying, political economy, legislatures, uncertainty}
%\end{quote}

\bigskip
\section{Introduction}
\label{sec:intro}

\subsection{Related Literature}
\label{sec:lit}

%I will begin by ... Section~\ref{sec:main} then ... In Section~ \ref{sec:example}, ... Section~\ref{sec:dis} explores .... Several extensions of the model are explored in Section~\ref{sec:ext} and Section~\ref{sec:concl} concludes.


\section{The Model}
\label{sec:model}

%If we're going to do an empirical implementation of $\ga$ below, model needs to start with a basic understanding of uncertainty, run through lobbying response to it, up through aggregation into $\ga$ and how lobbying behavior turns into a ``yea'' vote total.

\section{Some Theoretical Results}
\label{sec:res}

\section{Estimating Legislative Uncertainty}
\label{sec:est}

\section{An Application to the U.S. House of Representatives}
\label{sec:house}



%\subsection{Break Decision}

%\subsection{Uncertainty for Select Interest Groups}
%\label{sec:select}


\section{Conclusion}
\label{sec:concl}



\section*{Appendix}

			




\newpage
\section*{References}

\begin{list}{}{\setlength{\leftmargin}{0.3in}\setlength{\rightmargin}{0.0in}\setlength{\itemindent}{-0.3in}\setlength{\itemsep}{0.0in}}


\item Ansolabehere, S., J. de Figueiredo, and J. Snyder Jr. (2003), ``Why Is There so Little Money in U.S. Politics?,'' {\em Journal of Economic Perspectives}, 17, 105-130.

\item Becker, G., (1983) ``A Theory of Competition among Interest Groups for Political Influence.'' {\em Quarterly Journal of Economics} 98, 371-400.

\item Bombardini, M. (2008), ``Firm Heterogeneity and Lobby Participation,'' {\em Journal of International Economics}, 75, 329-348.

\item Bombardini, M., and F. Trebbi (2012): ``Competition and Political Organization: Together or Alone in Lobbying for Trade Policy?'' {\em Journal of International Economics}, 87, 18-26.

\item Clinton, J., Jackman, S., and D. Rivers, (2004): ``The Statistical Analysis of Roll Call Data.'' {\em American Political Science Review}, 98, 355-370.

\item Dal Bo, E. (2007): ``Bribing Voters.'' {\em American Journal of Political Science} 51, 789-803.

\item Dekel, E., Jackson, M., Wolinsky, A. (2005): {\em Vote Buying.} Unpublished manuscript, Caltech.

\item Gawande, K., P. Krishna and M. Robbins (2006): ``Foreign Lobbies and U.S. Trade Policy,'' {\em Review of Economics and Statistics}, 88, 563-571.

\item Goldberg, P. and G. Maggi (1999): ``Protection for Sale: An Empirical Investigation,'' {\em American Economic Review}, 89, 1135-1155.

\item Groseclose, T., Snyder, J. M. (1996): ``Buying Supermajorities.'' {\em American Political Science Review} 90, 303-315.

\item Grossman, G. and E. Helpman (1994): ``Protection for Sale,'' {\em The American Economic Review}, 84, 833-850.

\item Grossman, G. and E. Helpman (2005): ``A Protectionist Bias in Majoritarian Politics,'' {\em The Quarterly Journal of Economics}, 120, 1239-1282.

\item Henisz, W. and E. Mansfield (2006), ``Votes and Vetoes: The Political Determinants of Commercial Openness,'' {\em International Studies Quarterly}, 50, 189-211.

\item Kibris, A. (2012), ``Uncertainty and Ratification Failure,'' {\em Public Choice}, 150, 439-467.

\item Laffont, J., Tirole, J. (1994): ``A Theory of Incentives in Procurement and Regulation.'' Cambridge: MIT Press.

\item Laver, M. and K. Shepsle (1991), ``Divided Government: America is Not `Exceptional','' {\em Governance: An International Journal of Policy and Administration}, 4, 250-269.

\item Le Breton, M. and F. Salanie (2003), ``Lobbying under political uncertainty,'' {\em Journal of Public Economics}, 87, 2589-2610.

\item Le Breton, M. and V. Zaporozhets (2007), ``Legislative Lobbying under Political Uncertainty,'' Available at SSRN: \url{http://ssrn.com/abstract=1024686}.

\item Peltzman, S. (1976), ``Toward a More General Theory of Regulation.'' {\em Journal of Law and Economics} 19, 211-248.

\item Poole, K. T. (2005), ``Spatial Models of Parliamentary Voting.'' New York: Cambridge University Press.

\item Saiegh, S. (2009), ``Political  Prowess or Lady Luck? Evaluating Chief Executives' Legislative Success Rates,'' {\em The Journal of Politics}, 71, 1342-1356.

\item Saiegh, S. (2011) ``Ruling by Statute: How Uncertainty and Vote-Buying Shape Lawmaking.'' New York: Cambridge University Press.

\item Stigler, G. (1975) ``The Citizen and the State: Essays on Regulation.'' Chicago: Chicago University Press.


\end{list}

\end{document}